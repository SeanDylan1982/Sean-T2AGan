% !TeX program = xelatex
\documentclass{vilgym}

% General info
\title{Kunsti genereerimine teksti põhjal}
\authors{Karl-Joan Alesma, III MF}
\instructor{õp Malle Eglit}
\date{2019}


\addbibresource{viited.bib}

\begin{document}
    \maketitle
    \tableofcontents

    \unsection{Definitsioonid}
    \begin{description}
		\let\originalitem\item
		\renewcommand*{\item}[1][]{\originalitem[#1]\label{def:#1}}

        \item{GAN} ing.k. \textit{generative adversarial network}
    \end{description}

	\newcommand*{\seedefinition}[1]{(\hyperref[def:#1]{vt~definitsiooni})}

    \unsection{Sissejuhatus}
    

    Uurimistöö eesmärk on luua mudel, mis on võimeline ise genereerima kunsti teksti põhjal, ja et autor selle juures laiendadaks ja kinnitadaks oma teadmisi sügavõppest.

    Töö alguses annab autor ülevaate varasematest töödest selles valdkonnas ning tutvustab arenguid sügavõppes, mis on vajalikud töö mõistmiseks. See järel tutvustatakse erinevaid mudeleid, millele järgneb eksperimentaalne osa, kus analüüsitakse ning hinnatakse erinevate mudelite sooritust. 
    
    \section{Varasemad tööd}
    Uue materjali genereerimine on keeruline probleem. Kogu sügavõppes ajaloo vältelt on diskrimineerivad mudelid saavutanud paremaid tulemusi kui generatiivsed mudelid. Hiljuti on aga see muutunud GANide tekkega.

    GANid on saavutanud märkimisväärseid tulemusi pildi ja video generatsioonis ning ka super resolutsioonis. kunst

    tekst pildiks

    %  \section{Mudel}
    \section{Tehnilised detailid}

    \section{Eksperimendid}

    \unsection{Kokkuvõte}
    a
    \unsection{Summary}
    asd \parencite{cyclegan}

    % The bibliography
    \nocite{*} % List all entries, even when not cited
    \printbibliography[title={Kasutatud allikad}]

\end{document}
